\chapter{Introduzione}

\section{Obiettivi della tesi}

La tecnologia dei robot mobili autonomi e telecomandati è divenuta sempre più popolare per molteplici situazioni e obiettivi. I robot autonomi sono da tempo studiati per assistere nei compiti umani, sia nella vita di tutti i giorni, che in quella professionale~\cite{1,2}. Di recente, ad esempio, in ambito sanitario, sono stati studiati e realizzati robot autonomi che hanno permesso di fronteggiare l'emergenza Covid-19 nella sanificazione degli edifici sanitari~\cite{3}; nel contesto industriale, invece, vengono studiati metodi di pianificazione del percorso (\textit{planning path}) con prevenzione delle collisioni~\cite{4,5}. Pertanto, possiamo dire che i robot autonomi sono stati utilizzati principalmente in ambienti ben strutturati, per il raggiungimento di obiettivi quali il rilevamento del movimento, la pulizia all'interno di un edificio (come riportato in figura \ref{fig:robot-aspirapolvere}), o la produzione nelle catene di montaggio~\cite{6, 7}.
\begin{figure}[h]
	\centering
	\includegraphics[width=0.7\linewidth]{"images/robot aspirapolvere"}
	\caption{Immagine di un robot aspirapolvere di marca Thamtu.}
	\label{fig:robot-aspirapolvere}
\end{figure}
 
Tuttavia, sono diverse le situazioni in cui i robot autonomi non funzionano correttamente, come, ad esempio, in ambienti non ben strutturati che richiedono una certa \textit{flessibilità}. In questi casi, i robot telecomandati sono quelli che vengono maggiormente utilizzati, proprio perché i professionisti (o, comunque, chiunque ne comprenda il funzionamento) possono azionare e controllare a distanza il robot, ottenendo, per l'appunto, la sopracitata flessibilità richiesta~\cite{8, 9}.\\ 
In ambito sanitario, ad esempio, sono stati studiati robot telecomandati a supporto del personale ospedaliero~\cite{10, 11}, mentre in ambito industriale sono stati sviluppati algoritmi e robot telecomandati che riuscissero a perlustrare e tracciare ambienti più complessi~\cite{12}. Nella cura della salute e del benessere, essi sono stati utilizzati per assistere gli esseri umani nelle circostanze di ogni giorno; in situazioni di risposta ai disastri e in aree pericolose, questi sono stati ampiamente utilizzati per effettuare ispezioni o missioni di salvataggio~\cite{13, 14}.\\
I robot telecomandati ci permettono, dunque, di completare una serie di attività in base al giudizio e alla competenza dell'operatore. Per il sistema telecomandato vengono solitamente utilizzati due dispositivi: un monitor, che permette la visualizzazione di informazioni visive, e un dispositivo di controllo, un controller. L'utente aziona il robot telecomandato utilizzando il dispositivo di controllo mentre visiona il monitor, che visualizza le informazioni dai vari sensori. Affinché si possa operare in modo flessibile, sono necessari operatori esperti che comprendano e siano in grado di utilizzare al meglio il robot telecomandato. Per questo motivo, il funzionamento del sistema non è semplice e, soprattutto, non è esente da rischi, il che implica la possibilità di errori operativi. Questo perché è difficile per l'operatore comprendere le situazioni ambientali reali utilizzando un monitor che mostra solo le informazioni visive. Pertanto, per avere la giusta esperienza, sono necessarie numerose fasi di addestramento affinché l'utente riconosca l'ambiente circostante dalle informazioni visive, ma non sempre questo è possibile.\\
Nella robotica mobile, una delle caratteristiche maggiormente cercate è la capacità da parte del robot di riuscire a navigare in modo sicuro e allo stesso tempo fluido all'interno di un ambiente sconosciuto. Il problema, però, è che esso potrebbe incontrare ostacoli che possono essere rigidi o avere forme mutevoli. La loro traiettoria e la loro velocità possono essere soggette a cambiamenti imprevedibili nel tempo. Ed è proprio in questi casi che la pianificazione e la prevenzione sono strettamente limitate, proprio poiché richiederebbero una conoscenza preliminare dell'ambiente.\\
Questa tesi si concentra sullo sviluppo di un sistema che previene ed evita le collisioni. In particolare, per il raggiungimento di questi obiettivi, che, di fatto, aumentano notevolmente le prestazioni di sicurezza, è stato studiato nel dettaglio il cosiddetto ``\textit{force feedback}''.\\ Durante il funzionamento di un robot telecomandato, il force feedback viene spesso impiegato per assistere l’utente e migliorare la sua percezione degli ambienti e aumentare le sue capacità operative~\cite{15, 16, 17}. Perciò, la valutazione di un metodo combinato con l'assistenza di forze e gli aiuti visivi è uno studio molto significativo per i robot telecomandati. Per fare un parallelismo, in ambito automobilistico sono già in circolazione sistemi di prevenzione delle collisioni, come l'AEB (Autonomous Emergency Braking) o il FCW, (Forward Collision Warning), basati principalmente su una tecnologia a sensori ottici (LIDAR) che scansionano lo spazio nel contorno dell'automobile. Questi sistemi di assistenza alla guida sfruttano sensori di parcheggio, telecamera posteriore e radar per rilevare oggetti che si avvicinano all’auto durante le manovre e, in caso di rischio collisione, attiva i freni anche se il conducente non si è accorto dell’ostacolo. In un'indagine dell’HLDI (ente no profit), è stato analizzato l’impatto di questi sistemi sugli incidenti e le richieste di risarcimento. Viene mostrata di seguito in figura \ref{fig:incidenti-stradali-e-sistemi-di-assistenza-anti-collisioni} una tabella rappresentante l'indagine effettuata.
\begin{figure}[h]
	\centering
	\includegraphics[width=0.8\linewidth]{images/Incidenti-stradali-e-sistemi-di-assistenza-anti-collisioni}
	\caption{Impatto dei sistemi di assistenza anticollisioni sugli incidenti e sulle richieste di risarcimento.}
	\label{fig:incidenti-stradali-e-sistemi-di-assistenza-anti-collisioni}
\end{figure}\\Ebbene, secondo i dati elaborati dal HLDI le auto con AEB posteriore o frenata automatica in retromarcia, sono state coinvolte nel 28\% di richieste di risarcimento in meno e nel 10\% in meno di collisioni. Sembrerebbe scontata l'importanza di questi sistemi, eppure c’è da dire che la dotazione delle auto di massa è ancora ferma ai sensori di parcheggio e alle retrocamere.\\
Il sistema anticollisione per robot mobili completamente controllati da remoto proposto in questa tesi si basa non solo su laser, come i sistemi citati precedentemente per le automobili, ma anche e soprattutto sul concetto del force feedback: ciò vuol dire che il sistema non frenerà completamente il robot quando esso si avvicina agli ostacoli, ma lo farà rallentare in maniera esponenziale, modificando anche la traiettoria, in modo da evitare la collisione.\\ 
L'algoritmo proposto è stato valutato simulando un ambiente al chiuso utilizzando il meta-sistema operativo per robot \textbf{ROS}, acronimo che sta per \textbf{Robot Operating System}, e i suoi strumenti, di cui se ne parlerà meglio in seguito. I risultati sperimentali ottenuti utilizzando il metodo proposto mostrano che l'operabilità e le prestazioni di sicurezza sono particolarmente elevate. Durante lo sviluppo del progetto, è stato considerato come ambiente di simulazione l'edificio del DIAG, il Dipartimento di Ingegneria Informatica, Automatica e Gestionale ``Antonio Ruberti''.

\section{Organizzazione della tesi}

In questa sezione viene proposta un'illustrazione sulla struttura della tesi.\\
Nel \hyperref[cap:ROS]{Capitolo 2} verrà effettuato un excursus sulle tecnologie utilizzate, in particolare su ROS, sulla sua storia e sugli strumenti che mette a disposizione.\\ Nel \hyperref[descrizione]{Capitolo 3} verrà discusso nel dettaglio il sistema di anticollisione, l’algoritmo utilizzato, le istruzioni per eseguire il codice e testare il sistema, le scelte progettuali e le varie ragioni che hanno portato a queste scelte.\\
Nel \hyperref[esperimenti]{Capitolo 4} verranno esposti i vari esperimenti effettuati con i relativi risultati, e verranno mostrati anche alcuni dei limiti del sistema di anticollisione, che verranno poi ripresi nell'ultimo capitolo, il \hyperref[Conclusioni]{Capitolo 5}, in cui vengono illustrate le possibili situazioni dove è possibile sfruttare questo sistema come soluzione a molti problemi reali, e dove sono esposti possibili miglioramenti e implementazioni da aggiungere al sistema.